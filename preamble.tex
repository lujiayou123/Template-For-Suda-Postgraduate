%\usepackage{xeCJK}			% 中文支持
\usepackage{pdfpages}		% 插入pdf
\usepackage{graphicx}		% 图形支持
\usepackage{amssymb}		% 数学符号
\usepackage{amsmath}		% 数学符号
\usepackage{fancybox}		% 方框文字
\usepackage{wrapfig}		% 图片文字环绕
\usepackage{fancyhdr}		% 页眉设置
%\usepackage{cite}			% 引用文献
\usepackage{indentfirst}	% 首行缩进 
\usepackage{chngpage}
%\usepackage{letterAlinz}
\usepackage[colorlinks,linkcolor=blue,citecolor=blue]{hyperref}		% 让tableofcontents支持超链接
\usepackage[top=1in,bottom=1in,left=1.4in,right=1.2in]{geometry}	% 设置页边距
%\usepackage[top=0.8in,bottom=0.8in,left=1.2in,right=0.6in]{geometry}  %设置页边距(学校的要求)
%========================================
%		Settings
%\setCJKmainfont{SimSun}		% 字体设置,宋体
\setlength{\parindent}{2em}	% 首行缩进,2字符
%\setCJKmainfont{文泉驿正黑}

%========================================
%		Redefine commands 
%\renewcommand\abstractname{摘\ 要}		% 摘要 ,
%\renewcommand{\figurename}{图} 			% 图
%\renewcommand\refname{参考文献}			% 参考文献
%\renewcommand\contentsname{\centerline{目录}}	%目录居中
%\renewcommand{\today}{\number\year 年 \number\month 月 \number\day 日}	%中文日期

%========================================
%		Header Settings
\pagestyle{fancy}			% 
\chead{}	% 页眉中部
\lhead{基于任务推断的元强化学习}		% 页眉左部,设为空
\rhead{中文摘要}		% 页眉右部,设为空

%========================================