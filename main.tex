%\documentclass[UTF8]{ctexart}

%\begin{document}

%你好Latex!
%\heiti 你好
%\fontsize{20pt}{20pt}\songti 你好
%\end{document}

\documentclass[11pt,a4paper,UTF8]{ctexart}
%\usepackage{xeCJK}			% 中文支持
\usepackage{pdfpages}		% 插入pdf
\usepackage{graphicx}		% 图形支持
\usepackage{amssymb}		% 数学符号
\usepackage{amsmath}		% 数学符号
\usepackage{fancybox}		% 方框文字
\usepackage{wrapfig}		% 图片文字环绕
\usepackage{fancyhdr}		% 页眉设置
%\usepackage{cite}			% 引用文献
\usepackage{indentfirst}	% 首行缩进 
\usepackage{chngpage}
%\usepackage{letterAlinz}
\usepackage[colorlinks,linkcolor=blue,citecolor=blue]{hyperref}		% 让tableofcontents支持超链接
\usepackage[top=1in,bottom=1in,left=1.4in,right=1.2in]{geometry}	% 设置页边距
%\usepackage[top=0.8in,bottom=0.8in,left=1.2in,right=0.6in]{geometry}  %设置页边距(学校的要求)
%========================================
%		Settings
%\setCJKmainfont{SimSun}		% 字体设置,宋体
\setlength{\parindent}{2em}	% 首行缩进,2字符
%\setCJKmainfont{文泉驿正黑}

%========================================
%		Redefine commands 
%\renewcommand\abstractname{摘\ 要}		% 摘要 ,
%\renewcommand{\figurename}{图} 			% 图
%\renewcommand\refname{参考文献}			% 参考文献
%\renewcommand\contentsname{\centerline{目录}}	%目录居中
%\renewcommand{\today}{\number\year 年 \number\month 月 \number\day 日}	%中文日期

%========================================
%		Header Settings
\pagestyle{fancy}			% 
\chead{}	% 页眉中部
\lhead{基于任务推断的元强化学习}		% 页眉左部,设为空
\rhead{中文摘要}		% 页眉右部,设为空

%========================================

\begin{document}
%\begin{thebibliography}{99}	
	%================== 中英文摘要 ============
	% A 中英文摘要、关键词 
	\pagenumbering{Roman}
	%========================================
% 中文摘要
%\renewcommand\abstractname{基于任务推断的元强化学习}
%\begin{abstract}
%\begin{center}
%	\Large
%	\textbf{基于任务推断的元强化学习}\\
%	\Large
%	\textbf{摘要}\\
%\end{center}
%\ \ \ \  《狂人日记》是鲁迅创作的第一个短篇白话日记体小说,也是中国第一部现代白话文小说,写%于1918年4月。该文首发于1918年5月15日4卷5号的《新青年》月刊,后收入《呐喊》集,编入《鲁迅全%集》第一卷。
%\\
%\noindent\textbf{关键词:} 关键词1,关键词2,关键词3


\section*{基于任务推断的元强化学习}
\section*{摘要}
《狂人日记》是鲁迅创作的第一个短篇白话日记体小说,也是中国第一部现代白话文小说,写于1918年4月。该文首发于1918年5月15日4卷5号的《新青年》月刊,后收入《呐喊》集,编入《鲁迅全集》第一卷。
\\
\noindent\textbf{关键词:} 关键词1,关键词2,关键词3


\begin{adjustwidth}{25em}{0em}
	作者:\\
	指导老师:
\end{adjustwidth}
%\end{abstract}



%========================================
%		Header Settings
\pagestyle{fancy}			% 
\chead{}	% 页眉中部
\lhead{基于任务推断的元强化学习}		% 页眉左部,设为空
\rhead{摘要}		% 页眉右部,设为空

%========================================
	\newpage
	
	%========================================
% 英文摘要
%\begin{center}
%	\Large
%	\textbf{Meta-Reinforcement Learning with Task Inference}\\
%	\Large
%	\textbf{Abstract}\\
%\end{center}
%\ \ \ \  Diary of a madman is the first short diary style vernacular novel written by Lu Xun and the first modern vernacular novel in China. It was written in April 1918. This article was first published in the new youth monthly on May 15, 1918, Volume 4, No. 5, and then included in the collection of Nahuo and included in the first volume of Lu Xun's complete works.
%\\
%\noindent\textbf{Keywords:} Keyword1,Keyword2,Keyword3

\section*{Meta-Reinforcement Learning with Task Inference}
\section*{Abstract}
Diary of a madman is the first short diary style vernacular novel written by Lu Xun and the first modern vernacular novel in China. It was written in April 1918. This article was first published in the new youth monthly on May 15, 1918, Volume 4, No. 5, and then included in the collection of Nahuo and included in the first volume of Lu Xun's complete works.
\\
\noindent\textbf{Keywords:} Keyword1,Keyword2,Keyword3

\begin{adjustwidth}{25em}{0em}
	Written\ \ by:\\
	Supervised\ by:
\end{adjustwidth}
%\end{abstract}

%========================================
%		Header Settings
\pagestyle{fancy}			% 
\chead{}	% 页眉中部
\lhead{Meta-Reinforcement Learning with Task Inference}		% 页眉左部,设为空
\rhead{Abstract}		% 页眉右部,设为空

%========================================
	\newpage
	
	%================== 目录 =================
	% B 目录
	%\pagenumbering{roman}
	\thispagestyle{empty}
	\tableofcontents
	\thispagestyle{empty}
	%\setcounter{section}{1}
	\newpage
	
	%================== 引言  ================
	% C 引言
	
	\pagenumbering{arabic}
	\setcounter{section}{1}
\setcounter{subsection}{0}
\section*{第一章\ 引言}
\addcontentsline{toc}{section}{第一章\ 引言}
《狂人日记》是鲁迅创作的第一个短篇白话日记体小说,也是中国第一部现代白话文小说,写于1918年4月。该文首发于1918年5月15日4卷5号的《新青年》月刊,后收入《呐喊》集,编入《鲁迅全集》第一卷。
\subsection{研究背景及意义}
\subsubsection{研究背景}
\cite{mnih_playing_2013}研究背景\cite{rakelly_efficient_2019}
\subsubsection{研究意义}
研究意义
\subsection{研究现状}
研究现状
\subsection{研究内容}
研究内容
\subsection{论文组织结构}
论文组织结构

%========================================
%		Header Settings
\pagestyle{fancy}			% 
\chead{}	% 页眉中部
\lhead{基于任务推断的元强化学习}		% 页眉左部,设为空
\rhead{第一章\ 引言}		% 页眉右部,设为空

%========================================
	\newpage 
	
	%=================== 背景知识 =============
	% D 背景知识
	\setcounter{section}{2}
\setcounter{subsection}{0}
\section*{第二章\ 背景知识}
\addcontentsline{toc}{section}{第二章\ 背景知识}
引言
\subsection{研究背景及意义}
\subsubsection{研究背景}
研究背景
\subsubsection{研究意义}
研究意义
\subsection{研究现状}
研究现状
\subsection{研究内容}
研究内容
\subsection{论文组织结构}
论文组织结构

%========================================
%		Header Settings
\pagestyle{fancy}			% 
\chead{}	% 页眉中部
\lhead{基于任务推断的元强化学习}		% 页眉左部,设为空
\rhead{第二章\ 背景知识}		% 页眉右部,设为空

%======================================== % 
	\newpage 
	
	%=================== 第一个点 ===============
	% E 第一个点
	\setcounter{section}{3}
\setcounter{subsection}{0}
\section*{第三章\ 第一个点}
\addcontentsline{toc}{section}{第三章\ 第一个点望}
引言
\subsection{研究背景及意义}
\subsubsection{研究背景}
研究背景
\subsubsection{研究意义}
研究意义
\subsection{研究现状}
研究现状
\subsection{研究内容}
研究内容
\subsection{论文组织结构}
论文组织结构

%========================================
%		Header Settings
\pagestyle{fancy}			% 
\chead{}	% 页眉中部
\lhead{基于任务推断的元强化学习}		% 页眉左部,设为空
\rhead{第三章\ XXX}		% 页眉右部,设为空

%========================================
	\newpage
	
	%=================== 第二个点 ===============
	% F 第二个点
	\setcounter{section}{4}
\setcounter{subsection}{0}
\section*{第四章\ 第二个点}
\addcontentsline{toc}{section}{第四章\ 第二个点}
%\section{第四章\ 第二个点}
引言
\subsection{研究背景及意义}
\subsubsection{研究背景}
研究背景
\subsubsection{研究意义}
研究意义
\subsection{研究现状}
研究现状
\subsection{研究内容}
研究内容
\subsection{论文组织结构}
论文组织结构

%========================================
%		Header Settings
\pagestyle{fancy}			% 
\chead{}	% 页眉中部
\lhead{基于任务推断的元强化学习}		% 页眉左部,设为空
\rhead{第四章\ XXX}		% 页眉右部,设为空

%========================================
	\newpage
	
	%=================== 第三个点 ============
	% G 第三个点
	\input{third.tex}
	\newpage 
	
	%=================== 总结与展望 ============
	% H 总结与展望
	\setcounter{section}{6}
\setcounter{subsection}{0}
%\setcounter{secnumdepth}{0}
\section*{第六章\ 总结与展望}
\addcontentsline{toc}{section}{第六章\ 总结与展望}
引言
\subsection{研究背景及意义}
\subsubsection{研究背景}
研究背景
\subsubsection{研究意义}
研究意义
\subsection{研究现状}
研究现状
\subsection{研究内容}
研究内容
\subsection{论文组织结构}
论文组织结构

%========================================
%		Header Settings
\pagestyle{fancy}			% 
\chead{}	% 页眉中部
\lhead{基于任务推断的元强化学习}		% 页眉左部,设为空
\rhead{第六章\ 总结与展望}		% 页眉右部,设为空

%========================================
	\newpage 
	
	%==================== 参考文献 ===========
	% I 参考文献
	%\section{参考文献}
%========================================
%		Header Settings
\pagestyle{fancy}			% 
\chead{}	% 页眉中部
\lhead{基于任务推断的元强化学习}		% 页眉左部,设为空
\rhead{参考文献}		% 页眉右部,设为空

%========================================


	\pagestyle{fancy}			% 
	\chead{}	% 页眉中部
	\lhead{基于任务推断的元强化学习}		% 页眉左部,设为空
	\rhead{参考文献}		% 页眉右部,设为空
	\addcontentsline{toc}{section}{参考文献}	
	\bibliographystyle{IEEEtran} %参考文献样式
	\bibliography{references}    %bib文件    
	
	%\newpage 
	
	%==================== 致谢  =============
	% G 致谢
	%\input{acknowlegements.tex}
	%\newpage
	
	%==================== 附录 ===============
	% H 附录 (符号说明,原始材料等)
	%\input{appendix.tex}
%\end{thebibliography}
\end{document}